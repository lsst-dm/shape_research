%% Generated by Sphinx.
\def\sphinxdocclass{report}
\documentclass[letterpaper,10pt,english]{sphinxmanual}
\ifdefined\pdfpxdimen
   \let\sphinxpxdimen\pdfpxdimen\else\newdimen\sphinxpxdimen
\fi \sphinxpxdimen=.75bp\relax

\usepackage[utf8]{inputenc}
\ifdefined\DeclareUnicodeCharacter
 \ifdefined\DeclareUnicodeCharacterAsOptional
  \DeclareUnicodeCharacter{"00A0}{\nobreakspace}
  \DeclareUnicodeCharacter{"2500}{\sphinxunichar{2500}}
  \DeclareUnicodeCharacter{"2502}{\sphinxunichar{2502}}
  \DeclareUnicodeCharacter{"2514}{\sphinxunichar{2514}}
  \DeclareUnicodeCharacter{"251C}{\sphinxunichar{251C}}
  \DeclareUnicodeCharacter{"2572}{\textbackslash}
 \else
  \DeclareUnicodeCharacter{00A0}{\nobreakspace}
  \DeclareUnicodeCharacter{2500}{\sphinxunichar{2500}}
  \DeclareUnicodeCharacter{2502}{\sphinxunichar{2502}}
  \DeclareUnicodeCharacter{2514}{\sphinxunichar{2514}}
  \DeclareUnicodeCharacter{251C}{\sphinxunichar{251C}}
  \DeclareUnicodeCharacter{2572}{\textbackslash}
 \fi
\fi
\usepackage{cmap}
\usepackage[T1]{fontenc}
\usepackage{amsmath,amssymb,amstext}
\usepackage{babel}
\usepackage{times}
\usepackage[Bjarne]{fncychap}
\usepackage[dontkeepoldnames]{sphinx}

\usepackage{geometry}

% Include hyperref last.
\usepackage{hyperref}
% Fix anchor placement for figures with captions.
\usepackage{hypcap}% it must be loaded after hyperref.
% Set up styles of URL: it should be placed after hyperref.
\urlstyle{same}

\addto\captionsenglish{\renewcommand{\figurename}{Fig.}}
\addto\captionsenglish{\renewcommand{\tablename}{Table}}
\addto\captionsenglish{\renewcommand{\literalblockname}{Listing}}

\addto\captionsenglish{\renewcommand{\literalblockcontinuedname}{continued from previous page}}
\addto\captionsenglish{\renewcommand{\literalblockcontinuesname}{continues on next page}}

\addto\extrasenglish{\def\pageautorefname{page}}





\title{Likelyhood Of Shapes In Moment Space Documentation}
\date{Feb 20, 2018}
\release{}
\author{Nate B. Lust}
\newcommand{\sphinxlogo}{\vbox{}}
\renewcommand{\releasename}{Release}
\makeindex

\begin{document}

\maketitle
\sphinxtableofcontents
\phantomsection\label{\detokenize{algorithmWriteup::doc}}



\chapter{Weighted Moments}
\label{\detokenize{algorithmWriteup:construction-of-a-likelyhood-function-for-the-moments-of-an-image}}\label{\detokenize{algorithmWriteup:weighted-moments}}
Define the weighted moments as measured from the image. Let \(r_i\) be a
vector \([x_i,y_i]\).
\begin{equation*}
\begin{split}R_0 = \sum_i z(r_i)w(r_i)\end{split}
\end{equation*}\phantomsection\label{\detokenize{algorithmWriteup:equation-first_raw_moment}}\begin{equation}\label{equation:algorithmWriteup:first_raw_moment}
\begin{split}R_1 = \sum_i z(r_i)w(r_i)r_i \quad [1]\end{split}
\end{equation}\phantomsection\label{\detokenize{algorithmWriteup:equation-second_raw_moment}}\begin{equation}\label{equation:algorithmWriteup:second_raw_moment}
\begin{split}R_2 = \sum_i z(r_i)W(r_i)r_ir_i^T \quad (3)\end{split}
\end{equation}

\chapter{Normalized Weighted Moments}
\label{\detokenize{algorithmWriteup:normalized-weighted-moments}}
The normalized weighted moments can be calculated as follows:

\phantomsection\label{\detokenize{algorithmWriteup:equation-zeroth_norm_moment}}\begin{equation}\label{equation:algorithmWriteup:zeroth_norm_moment}
\begin{split}M_0 = R_0 \quad\end{split}
\end{equation}\phantomsection\label{\detokenize{algorithmWriteup:equation-first_norm_moment}}\begin{equation}\label{equation:algorithmWriteup:first_norm_moment}
\begin{split}M_1 = \frac{R_1}{M_0} \quad(5)\end{split}
\end{equation}\phantomsection\label{\detokenize{algorithmWriteup:equation-second_norm_moment}}\begin{equation}\label{equation:algorithmWriteup:second_norm_moment}
\begin{split}M_ 2 = \frac{R_2}{M_0} - M_1M_1^T \quad\end{split}
\end{equation}

\chapter{Weight Function}
\label{\detokenize{algorithmWriteup:weight-function}}
We will define the weight function used in calculating moments to be an
elliptical Gaussian:
\begin{align*}\!\begin{aligned}
w(r_i) = \frac{e^{-\frac{1}{2}r_iC^{-1}_2r_i^T}}{w_0}\\
w_0 = \int_{-\infty}^{\infty}e^{-\frac{1}{2}rC^{-1}_2r^T}d^kr\\
\end{aligned}\end{align*}

\section{Debiasing}
\label{\detokenize{algorithmWriteup:debiasing}}
The weight function used in calculating the moments is useful for suppressing
extranious contributions from noise, but it also biases the measurement. To
correct for this, we can calculate an approximate debiasing factor by assuming
the source \(z(r_i)\) is itself a Gaussian with moments \(Q_i\) of the
form:
\begin{align*}\!\begin{aligned}
z(r_i) = \frac{e^{-\frac{1}{2}(r_i - Q_1) Q_2^{-1}(r_i-Q_1)^T}}{z_0}\\
z_0(Q_2) = \int_{-\infty}^{\infty}e^{-1\frac{1}{2}rQ^{-1}_2r^T}d^kr\\
\end{aligned}\end{align*}

\subsection{Zeroth Moment}
\label{\detokenize{algorithmWriteup:zeroth-moment}}
With this assumption the calculation of \(R_0\) becomes:
\begin{equation*}
\begin{split}R_0 = \frac{1}{W_0}\frac{Q_0}{z_0(Q_2)}\int_{-\infty}^{\infty}
      e^{-\frac{1}{2}(r-Q_1)Q_2^{-1}(r-Q_1)^T} e^{-\frac{1}{2}(r-C_1)C_2^{-1}
      (r-C_1)^T} d^kr\end{split}
\end{equation*}
Using the Matrix cook book, we recognize the integral can be re-expressed as:
\begin{align*}\!\begin{aligned}
= \frac{1}{N}\int_{-\infty}^{\infty}e^{-\frac{1}{2}(r-\alpha)\beta^-1
                                       (r-\alpha)^T}d^kr\\
N = \sqrt{\det(2\pi(Q_2^{-1} + C_2^{-1}))}\\
\alpha = (Q_2^{-1} + C_2^{-1})^{-1}(Q_2^{-1}Q_1 + C_2^{-1}C_1)\\
\beta = (Q_2^{-1} + C_2^{-1})^{-1}\\
\end{aligned}\end{align*}
If we make the assumption that \(\beta_x\beta_y - \beta_{xy}^2 > 0\) the
above integral evaluates to:
\begin{equation*}
\begin{split}= \frac{1}{N}\frac{2 \, \sqrt{2} \sqrt{\frac{1}{2}} \pi}{\sqrt{\beta_{x}}
  \sqrt{-\frac{\beta_{\mathit{xy}}^{2} - \beta_{x} \beta_{y}}{\beta_{x}}}}\end{split}
\end{equation*}

\subsection{First Moment}
\label{\detokenize{algorithmWriteup:first-moment}}
see {\hyperref[\detokenize{algorithmWriteup:first_raw_moment}]{\emph{eq 1}}}



\renewcommand{\indexname}{Index}
\printindex
\end{document}